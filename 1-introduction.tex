\chapter{Introduction}

Ranking can informally be defined as ``a relationship between a set of items such that, for any two items, the first is either 'ranked higher than', 'ranked lower than' or 'ranked equal to' the second''\footnote{\hhref{https://en.wikipedia.org/wiki/Ranking}}.
\todo{Assigning numbers}

Ranking in computer games is mainly done to reward skilled or lucky players. Introducing a dimension of competetion makes the game more exciting and hopefully -- from the game makers point of view -- more profitable. This can be done by creating a leaderbord that shows the highest scores or by giving some measure of improvement to the player, eg. number of ranks advanced by the last game. Other reasons for ranking skills in computer games include gambling and matching equally skilled players for challenges.

As a player of a computer game you want to know if you are making progress in your gaming. However, games that can be considered having even a moderate success on todays marketplaces have at least hundreds of thousands of players not knowing each other. Therefore, cheating by approximating ranks works perfectly fine most of the time. Also, approximated ranks come pretty close to the real rank as we shall see. In some cases though an exact rank needs to be calculated. Obviously it would be embarrassing if two players were estimated to have rank number one. In general, higher ranks calls for greater precision.

\section{Ranking, a definition}

When ranking is mentioned in scientific writings it is usually in the sense of assigning a weight to an item in a set from a function of properties of that item. A search on a price comparison site for ``red car'' would assign a high weight to items that are cars and also happens to be red. Eventually a list of red cars will be presented.

When the word \emph{ranking} appears in the rest of this thesis it means the procedure of getting the position for an item with a weight in a list sorted on the weight of the item -- not how that specific weight is decided upon. Or to be precise; the procedure of getting the position for a user with a highscore from a large number of highscores in a database, indexed on the highscore.

It is unfortunate that there is no commonly used, distinct word for \emph{ranking}. However, it should be clear from the context which definition is referred.

\section{Problem description, outline and methods}

The overall goal is to investigate how to get the rank, exact or an approximation, for a new highscore in the context of an online computer game. Doing ranking of highscores by counting the number of highscores better than a specific score is not a feasible option. A tree-based approach to counting seems promising but have a number of drawbacks that will be discussed in section \ref{sec:tree}. Two methods based on linear interpolation will be presented: \emph{Buckets with Global Query} and a method relying on \emph{Frugal streaming} (\cref{bucket,frugal} will be discussed, the latter one for demonstrative purposes. Finally, a variant of the \emph{Buckets with Global Query}-algorithm with dynamic buckets will be presented and empirically evaluated in \cref{method,results,discussion}.

The questions to be researched in this thesis are

\begin{enumerate}
  \setlength\itemsep{-0.4em}
\item How to get a rank for a score efficiently and in near real time when the set of scores is large?
\item What is the performance characteristics of Bucket with Global Query algorithm when adapting the buckets to changes in the underlying highscore distribution?
\end{enumerate}

To attend to the first question, a review of the ranking approaches mentioned above will be provided in \cref{ranking-algorithms}, starting by a naive approach derived from the definition of ranking, ending with a draft for a streaming and approximating approach. The algorithms presented in chapter \cref{ranking-algorithms} do certainly not represent a complete listing of all possible ways of solving the problem stated but will nevertheless provide some insight in the area.

To answer the second question on the performance characteristics of the Buckets with Global Query with dynamic buckets, a system for evaluating the performance in terms of efficiency and precision is built. The precise method for this evaluation is left for \cref{method}.
