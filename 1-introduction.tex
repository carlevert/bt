\chapter{Introduction}

\section{Background}

\begin{shaded}
  Define ranking. Citation from Wikipedia, need better source.
\end{shaded}
``A ranking is a relationship between a set of items such that, for any two items, the first is either 'ranked higher than', 'ranked lower than' or 'ranked equal to' the second''

\begin{shaded}
  Typical CS related ranking scenarios - AI, search engines
\end{shaded}
 



\begin{shaded}
  Scoring function - How to produce ordinal numbers 
\end{shaded}

\subsection{Ranking in online games}

\begin{shaded}
Define what ranking is in the context of games. Ie what to rank time or skills? In case of skills - same skills may not translate into same score (as in TrueSkill). 

\end{shaded}

\textbf{Determine winning player or creating a leaderboard}

Most games have some kind of skill based ranking system in order to make the game interesting. Usually the skills are expressed as a score, ie a number.


Higher score \ra better

TrueSkill (scoring function) \ra score

time \ra lower or higher time is better

\textbf{Matching players}
The score that players get in a game may also be used to determine winners of tournaments, for gambling, for pairing similarily skilled players in a matchmaking process and so on.

\textbf{Approximate ranks are OK in many cases} Approximate matches of queries are commonplace in the text world (Top-k Selection Queries over Relational
Databases: Mapping Strategies and
Performance Evaluation)

General requirements (ie. an exact solution may not be needed all the time - may be crucial among the highest scores, when competing and if there is some gambling.
  
\section{Problem description}

\begin{shaded}The goal with this section is to

  1) clearify why this seemingly trivial problem is a real problem (O(n) \ra not feasible

  2) Something about a a very common business model (small startup, free-to-play) limited budget and business not large enough to run server park on its own. \end{shaded}

How to get a live rank for a score when set of scores are large

\subsection{Infrastructure}

\begin{shaded}
  Running applications on Platform-as-a-Service-services have implications on the application to be run.

  Performance-related-issues, cogestion.

  Pricing.

  \end{shaded}

Google App Engine (often referred to as GAE or simply App Engine) is a platform as a service (PaaS) cloud computing platform for developing and hosting web applications in Google-managed data centers. Applications are sandboxed and run across multiple servers.

Java, other languages
 
Other services (memory cache)

\textbf{App Engine Datastore} is a schemaless NoSQL datastore providing robust, scalable storage for your web application.

\subsubsection{Pricing}

Price model
