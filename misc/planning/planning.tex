\documentclass[10pt, titlepage, oneside, a4paper]{article}

\usepackage[T1]{fontenc}
\usepackage[english]{babel}
\usepackage{amssymb, graphicx, fancyheadings}
\usepackage[utf8]{inputenc}
\usepackage{wrapfig}

\addtolength{\textheight}{42mm}
\addtolength{\voffset}{-22mm}

\setlength{\parindent}{0pt}
\setlength{\parskip}{10pt}

\begin{document}
\begin{center}
\begin{large}
  5DV129: Examensarbete för\\ kandidatexamen i datavetenskap, 15.0 hp
  
\vspace{8mm}
\begin{LARGE}\textbf{Planning}\end{LARGE}
\vspace{6mm}

\end{large}

Carl-Evert Kangas\\\texttt{dv14cks@cs.umu.se}\\March 31, 2016
\vspace{3mm}

\end{center} 

\section*{Background}
 
A company making games have noticed that data integrity in some rare
cases is compromised in the back end systems. These systems are web
applications written in Java and hosted by Google on their App Engine
platform. Data is stored in the so called Datastore which is a
schemaless, NoSQL database. A memory cache, also provided by the App
Engine platform is used for speed and to keep costs down. An open
source library called Objectify is used for convenient access to the datastore.

\section*{Problem description}

The customer have observed entities where data integrity is
compromised. An object \{ dead: false, alive: true \} describing the
state of a cat seems in rare cases end up having impossible states
like \{ dead: false, alive: false \}\footnote{The actual objects are
  both larger and more complex}.

Updates to the datastore are transactional so the problem probably
lies somewhere else. One hypothesis is that the cache is updated
field-by-field and that the cache update may be interrupted before it
is finnished. Another possible explanation to the behaviour could be
related to the fact that values may be evicted from the cache if the
cache is low on memory.

The questions to be researched in the thesis are as follows:

\textbf{1. Can the alleged observations be reproduced?}\\
An essential question is whether this problem can be reproduced in a
systematic way that support the hypothesis. While data integrity
compromised entities\footnote{Data objects in the Datastore are known
  as \emph{entities}} have been observed, the cause of the problem may
be something completley different than discussed above.

\textbf{2. Describe the implementation of Objectify from the
  perspective on how transactions are handled}\\ The notion of transactions seems to be used mostly in the context of
databases. In this case, the database itself claims to handle
transactions. Can the concept of trasactions be extended to include
the caching service and what would this mean in terms of additional
computations and database accesses?

\textbf{3. Which apparent solutions could be devised?}\\
Could some well known ways of performing transactions be reused in
this context? In case the patterns aren't applicable, then why not?


\section*{Methods}

Since the companys issues comes from a live system it’s data cannot
be used for experiments. Conequently, to adress the first question a
small system will be built to research the problem. A positive
demonstration will acknowledge the problem. There are some obvious
variables that needs to be included in the experiment: the execution time (the execution
is shut down after 30 seconds in the platform) and the size of the
objects to be stored. A more specific method for performing the experiment
and handling the statistics will be discussed in the actual thesis.

The second question appears to be a litterature study of ways of doing
transactions. While not a very big area in itself when only classical
databases are concerned, I am looking forward to see what studies
there are on globally distributed databases.

Lastly, can a regular transaction-pattern be applied to make Objectify
transactional? This may or may not be doable within the given
timeframe. If a complete solution appears to be out of reach, then at
least some conclusions regarding such an implementation from the study
described above should be drawn. 

\section*{Other aspects and clarifications}

\subsection*{Ethics and social considerations}

None.

\subsection*{Regarding the company mentioned}

I am not in any way formally associated with the company mentioned. I
am not so to speak ''doing my thesis there'' and I'm not formally
supervised by them.

This is an issue that needs to be addressed somehow. Either by ignoring
the source of the problem (and lose some context) or formally make
an agreement about referring to their enterprise. 

\subsection*{Main focus}

The main focus is on the Objectify library. There are alternative
API:s for accessing datastore as well as the native API. A full review
of them does not seem to add valuable information to the questions as
they are formulated at the moment. Still, a brief review of the most
common alternatives could be done in the Discussion chapter.

\newpage

\section*{Planning}

It is roughly seven weeks till the nearly completed manuscript should
be handed in at May 20. Making a complete and meaningful plan for the
whole project is not possible, so I have deliberatly not put any
effort into planning the last weeks.

\textbf{Week 14, 4/4--8/4}\\
High priority to making a somewhat working program that can be used
for answering question one and design the experiment. Write an introdction and read up on relevant things.

\textbf{Week 15, 11/4--15/4}\\
Need to get into how to deal with question two, i.e. what kinds of
notions to use. Contact developers working with Objectify regarding
their design choices. Write method-chapter. Introduction should be finalized.


\textbf{Week 16, 18/4--22/4}\\
Since the mid seminar is scheduled the monday after week 16, at least 
questions 1 and 2 stated in the problem description earlier needs to either
have an answer or a defined type of answer. Otherwise they
need to be reformulated urgently. Method-chapter should be more or
less final.




\subsection*{Litterature}

There are loads of litterature regarding databases and my starting
point is the textbooks used at the database courses held by the
department: Henry F. Korth -- Database System Concepts and Elmarsi
Ramex -- Fundamentals of Database Systems. The latter cover in its
latest edition both distributed databases and NoSQL systems.

Also there seems to exist a lot of high quality papers by
major companies in this field.

The source code for Objectify seems well documented and should
investigated and some contact with the community around it needs to be established.

\subsection*{Typesetting and source code}

I'm planning to write the thesis in LaTeX mainly because of the BibTeX
reference management. A public Git repository at
\texttt{https://github.com/carlevert/bt} has been set up. The
repository so far contains mostly boilerplate LaTeX, but will later be
accompanied by source code for the projects.

\section*{Risks and risk management}

One obvious risk is that this project fails at the first question,
i.e. it's not possible to demonstrate the inconsitencies. To make the
impact of this happening a demonstration of the problem is of top
priority, that is, scheduled as soon as possible.

Another risk is that the theory regarding this problem is too
straightforward. While not an optimal solution because it may cause
more breadth than depth is extending the area of
research can be a solution. A total shift in focus is the ultimate but
unwanted solution.

\end{document}
