\documentclass[10pt, titlepage, oneside, a4paper]{article}

\usepackage[T1]{fontenc}
\usepackage[swedish]{babel}
\usepackage{amssymb, graphicx, fancyheadings}
\usepackage[utf8]{inputenc}
\usepackage{wrapfig}

\addtolength{\textheight}{42mm}
\addtolength{\voffset}{-22mm}

\setlength{\parindent}{0pt}
\setlength{\parskip}{10pt}

\begin{document}
\begin{center}
\begin{large}
5DV129: Examensarbete för kandidatexamen i datavetenskap, 15.0 hp

\textbf{Planning}
\end{large}

Carl-Evert Kangas, 801202-2995, \texttt{dv14cks@cs.umu.se}, \today

\end{center}

\thispagestyle{empty}
%% \subsection*{Beskriv området personen arbetat inom}

Vinton Cerf, född 23 juni 1943, är en internetpionjär och kallas ofta Internets fader (tillsammans med Robert Kahn, medmottagare av Turingpriset 2004).

Efter att ha avlagt motsvarande en kandidatexamen i matematik på Stanford University arbetade Cerf två år på IBM som ingengör. Detta väckte intresset för datorvetenskap och Cerf lämnade IBM för studier vid UCLA (University of California, Los Angeles) där han tog en masterexamen 1970 och doktorerade 1972 med avhandlingen \textit{Multiprocessors, Semaphores, and a Graph Model of Computation}.

Under sin tid som doktorand arbetade Cerf i professor Leonard Kleinrocks \textit{data packet networking group} som ansvarade för en av de fyra första noderna i ARPANET. Det var här som Cerf blev involverad i diskussionerna kring hur mjukvaran i ARPANET-värdarna skulle se ut. Efter att ha doktorerat blev Cerf forskarassistent på Stanford 1972-76 där han forskade om paketförmedlade nätverk och designade TCP (Transmission Control Program) tillsammans med Kahn.

1978 delade Cerf (m.fl.) TCP i två delar. Den del av TCP som ansvarade för adressering och möjliggör routing blev IP och den del av protokollet som ansvarade för att ordna paket och garantera leverans etc blev TCP (Transmission Control Protocol).

\vspace{7mm}

%% \subsection*{Nämn, det du uppfattat vara dennes viktigaste bidrag}

Cerfs viktigaste bidrag till världshistorien är utan tvekan att han lagt grunden till dagens TCP/IP och därmed skapat förutsättningarna för den nya värld som internet innebär.
Detta framgår även av motiveringen till Turingpriset, även om man inte spekulerar i konsekvenserna av hans arbete:

\begin{quotation}
Vinton G. Cerf and Robert E. Kahn led the design and implementation of the Transmission Control Protocol and Internet Protocol (TCP/IP) that are the basis for the current internet. They formulated fundamental design principles of networking, specified TCP/IP to meet these requirements, prototyped TCP/IP, and coordinated several early TCP/IP implementations.
\end{quotation}

Cerf är noga med att poängtera att han inte ensam har skapat internet och framhåller ofta Robert Kahn som lika framstående. I en intervju i ett program från Vetandets värld i P1\footnote{http://t.sr.se/1w6hOKj} om Vint Cerf hävdar Patrik Fältström att Cerf var den som under 1980-talet övertalade världens regeringar och institutioner om att internet var något att satsa på.

\vspace{7mm}

%% \subsection*{Redogör för de problem- och frågeställningar som personen har fått priset för}

De problem och frågeställningar som Cerf (och Kahn) bidragit till att lösa är i stort sett de problem som TCP och IP-protokollen hanterar, som exempel kan nämnas adressering och felkontroll.

Ett annat viktigt bidrag är att Cerf i och med det som slutligen blev TCP/IP lyfter ansvaret för adressering och transport från olika nätverk till de värdar som implementerar protokollen.
 
\end{document}